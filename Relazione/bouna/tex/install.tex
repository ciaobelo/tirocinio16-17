Per installare Docker su qualsiasi altra sistema *unix o win seguire la guida ufficiale di Docker \href{https://docs.docker.com/engine/installation/}{https://docs.docker.com/engine/installation/}.
In questo sezione vi mostrerò la procedura per l'installazione dello strumento Docker CE (comiunity edition) su una macchina Ubuntu e come verificare se la procedura sia andata a buon fine. Al fine di non riscontrare problemifuturi è necessario verificare che la versione appena installata sia un varsione maggiore o uguale 1.13.\\Poichè nelle precedenti versioni non sono embedded in docker alcuni tool di automazione e la piattaforma di container scheduling swarmdi cui parleremo più tardi.
\subsubsection {Prerequisiti minimi}Per installare Docker CE è \underline{necessario} avere installato una delle seguenti distribuzioni ubuntu a 64bit in altro casi guarda il link della documentazione ufficiale:
\begin{itemize}
	\item Yakkety 16.10
	\item Xenial 16.04
	\item Trusty 14.04
\end{itemize}
\textbf{1. Step}
\begin{verbatim}

   $ which curl

#  If curl isn’t installed, install it after updating your manager:

   $ sudo apt-get update
   
   $ sudo apt-get install curl

#  Get the latest Docker package.

   $ curl -fsSL https://get.docker.com/ | sh

#  The system prompts you for your sudo password. Then, it downloads and installs Docker and its dependencies.

# Note: If your company is behind a filtering proxy, you may find that the apt-key command fails for the Docker repo
# during installation. To work around this, add the key directly using the following:

   $ curl -fsSL https://get.docker.com/gpg | sudo apt-key add -
     
\end{verbatim}
*IN ADDITION*\\In aggiunta se si vuole si può considerare l'opzione di aggiungere il programma docker alla lista dei programmi con privilegi elevati in modo da poterlo eseguire in modalità non root o senza il comando \texttt{sudo} ogni volta:
\begin{verbatim}

   $	sudo usermod -aG docker <name_of_the_user_to_add>
  	
\end{verbatim}  
Per esempio se si vuole aggiungere l'utente corrente il comando andrà così formattato:
\begin{verbatim}
  
   $  sudo usermod -aG docker `whoami`
    
\end{verbatim}
Ricordati che le modifiche diventeranno effettive solo dopo il logout!\\\\
*PACKAGE EXTRA RACCOMANDATI*\\
L'installazione di questi package extra è necessaria se si vuole montare aufs fisibile anche su docker.
\begin{verbatim}

	$ sudo apt-get update
	
	$ sudo apt-get install curl \
	    linux-image-extra-$(uname -r) \
	    linux-image-extra-virtual

\end{verbatim}
DOPO L'INSTALLAZIONE:\\
procedo con la verifica dell'installazione.\\
procedere con esercizoio di base.\\
Verifica che il servizio docker si in essecuzione sulla tua maccchina ad moogni modo per  sicurezza eseguire la seguente sequenza di comandi:
\begin{verbatim}

	$ sudo service docker status

#next command restart all docker service

	$ sudo service docker restart

\end{verbatim}
In caso arrivati a questo punto per qualche motivo l'istallazine non sia andata a buon fine si potrebbe provare con questa sequenza di comandi:\\
1. Set up del repository\\
Aggiunta di Docker CE repository in Ubuntu. Tramite il seguente comando \texttt{lsb\_release -cs} stampo il nome della mia versione corrente di Ubuntu, esempio "xenial or trusty".
\begin{verbatim}

sudo apt-get -y install \
apt-transport-https \
ca-certificates \
curl

curl -fsSL https://download.docker.com/linux/ubuntu/gpg | sudo apt-key add -

sudo add-apt-repository \
"deb [arch=amd64] https://download.docker.com/linux/ubuntu \
$(lsb_release -cs) \
stable"

sudo apt-get update
\begin{verbatim}
2. Get Docker CE

Install the latest version of Docker CE on Ubuntu:


sudo apt-get -y install docker-ce

\end{verbatim}
*Tratto direttamente dalla documentazione originale (causa recenti aggiornamenti della piattaforma è possibile che alcuni comandi per alcune procedure siano variati nel frattempo invito sempre a verificare sulla documentazione).
\subsubsection{Verifica dell'installazione:}Ora dovremmo essere in grado a lanciare il comando \texttt{doocker run hello-world} e dovremmo ottenere un output tipo questo:
\begin{verbatim}

	$ docker run hello-world
	
	Hello from Docker!
	This message shows that your installation appears to be working correctly.
	
	To generate this message, Docker took the following steps:
	...(snipped)...
	
\end{verbatim}
Iinfine per motivi di compatibilità e per evitare problemi con l'esecuzione di Swarm in docker sucessivamente, accertiamoci di che la versione sia uguale o superiore alla 13.0 eseguendo  il comando \texttt{docker --version} e se vogliamo maggiori informazioni ppossiamo digitare il comando \texttt{docker info}.
\begin{verbatim}

	$ docker --version
	Docker version 17.05.0-ce-rc1, build 2878a85

\end{verbatim}
Arrivati a questo punto si è pronti per iniziare a sperimentare passsando alla sezione dove creeremo di persona le nostre immagini.

