\textbf{Kubernetes}\\
Kubernetes è un container scheduler reso disponibile come open source da Google, che afferma di utilizzare i container da circa dieci anni. Kubernetes, in tal senso, rappresenta un modo per condividere con il resto del mondo l’esperienza accumulata in questo periodo.\\
Kubernetes introduce il concetto di pod. I pod sono una serie di container che devono essere collocati insieme nello stesso host. Esistono svariate situazioni o modelli ricorrenti in cui una funzione come questa può essere necessaria: per esempio, un’applicazione potrebbe aver bisogno di un’altra applicazione “di accompagnamento” che si occupi di inviare i log \href{http://blog.kubernetes.io/2015/06/the-distributed-system-toolkit-patterns.html}{pagina del blog in riferimento}.\\
Kubernetes introduce inoltre il concetto di “controller di replicazione”. Il compito di replication controller consiste nel garantire che tutti i pod di un cluster siano “in salute”. Se un pod è giù o non sta fornendo le prestazioni attese, il replication controller lo rimuoverà dal cluster e creerà un nuovo pod.\\
Oltre a questo, Kubernetes cerca di essere portabile tra implementazioni private di Docker e implementazioni cloud pubbliche, grazie all’uso di una architettura a plugin. Per esempio, la configurazione del load balancer necessaria per la definizione di un servizio attiverà un plugin differente a seconda che il deploy del progetto sia effettuato in Google Cloud invece che in Amazon Web Service. L’intenzione è quella di aggiungere plugin per collegarsi a diversi tipi di storage, per effettuare diverse tipologie di deployment, per utilizzare fornitori di sicurezza diversi e così via.\\