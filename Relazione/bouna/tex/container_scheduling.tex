\textbf{Container scheduling}:\\
Il container scheduling (più o meno “calendarizzazione dei container”) è la possibilità di definire e attivare un ambiente potenzialmente composto di diversi container. Come già detto, è un concetto simile a quello dell’orchestrazione negli ambienti cloud basati su Virtual Machine.\\
Attraverso il container scheduling si dovrebbe riuscire a definire la corretta sequenza di attivazione dei container, la dimensione del cluster di container che è necessario creare e la qualità del servizio dei container: per esempio, un’app potrebbe richiedere un minimo garantito di di memoria e di accessi al disco.\\
Tool avanzati per l’orchestrazione devono garantire anche che un determinato ambiente rifletta lo stato desiderato dopo essere stato creato: significa che questi strumenti di orchestrazione possono anche svolgere un certo grado di monitoraggio ed effettuare, almeno in parte, una autoriparazione nel caso lo stato dell’ambiente non si allinei con quello stabilito dalla configurazione.\\
Un’altra importante caratteristica comune a molti scheduler per container è la capacità di definire un livello di astrazione intorno a un cluster di container identici, in maniera tale che essi siano visti come servizio, vale a dire un indirizzo IP e una porta. Questo risultato è ottenuto facendo impostare allo scheduler un load balancer davanti al cluster di container accompagnato dalle regole di indirizzamento necessarie.\\
In quest’ambito, alcuni degli strumenti disponibili sono Swarm e e Docker Compose, Apache Mesos, Kubernetes. Li vediamo di seguito.